\documentclass[electronics,article,submit,pdftex,moreauthors]{Definitions/mdpi} 
%=================================================================
% MDPI internal commands - do not modify
\firstpage{1} 
\makeatletter 
\setcounter{page}{\@firstpage} 
\makeatother
\pubvolume{1}
\issuenum{1}
\articlenumber{0}
\pubyear{2025}
\copyrightyear{2025}
%\externaleditor{Firstname Lastname} % More than 1 editor, please add `` and '' before the last editor name
\datereceived{ } 
\daterevised{ } % Comment out if no revised date
\dateaccepted{ } 
\datepublished{ } 
%\datecorrected{} % For corrected papers: "Corrected: XXX" date in the original paper.
%\dateretracted{} % For retracted papers: "Retracted: XXX" date in the original paper.
\hreflink{https://doi.org/} % If needed use \linebreak
%\doinum{}
%\pdfoutput=1 % Uncommented for upload to arXiv.org
%\CorrStatement{yes}  % For updates
%\longauthorlist{yes} % For many authors that exceed the left citation part
%\IsAssociation{yes} % For association journals

%=================================================================
\usepackage{graphicx}
%=================================================================

\Title{Broadband RF Phased Array Design with MEEP: A 3D-Printed Open-Source RF Horn in the multi-GHz Bandwidth}
\TitleCitation{Title}
\newcommand{\orcidauthorA}{0000-0003-4055-4553}
\Author{Jordan C. Hanson $^{1}$\orcidA{}*, Adam Wildanger$^{1}$}
\AuthorNames{Jordan C. Hanson, Adam Wildanger}
\AuthorCitation{Hanson, J. C.; Wildanger, A.}
\address[1]{$^{1}$ \quad Whittier College, Whittier, CA, USA (www.whittier.edu)}
\corres{Correspondence: jhanson2@whittier.edu}

\abstract{Radio-frequency (RF) antenna design traditionally proceeds in three phases.  First, the design performance is modeled with expensive and proprietary computational electromagnetism (CEM) software.  Second, the design is fabricated using intricate metal machining.  Third, the fabricated design is characterized using benchtop RF measurement tools.  The traditional process can include machine learning algorithms for optimization.  We have developed an open-source alternative process that utilizes the MIT Electromagnetic Equation Propagator (MEEP) CEM package for design and simulation, and 3D printing with conductive filament for fabrication.  Using this process, we designed and fabricated an exponentially curved RF horn antenna.  To characterize our design, we show that the E-plane and H-plane radiation patterns, the VSWR, and the cross-polarization ratios match our CEM calculations.  These results indicate that our design is a linearly polarized, broadband RF horn antenna in the [5.5-6] GHz bandwidth.  The bandwidth of our instrumenation is limited to 6 GHz, and our CEM calculations predict good performance above 6 GHz.  We therefore conclude the bandwidth of our printed design extends above 6 GHz.  Future work will include expanding to lower bandwidth by 3D printing larger antennas, and constructing a broadband RF phased array.}

% Keywords
\keyword{Computational Electromagnetism (CEM), Additive Manufacturing, MEEP, \\ RF Engineering, Open-Source Design}

\begin{document}

\section{Introduction}

Broadband RF antennas are ubiquitous tools within scientific instrumentation and communication applications.  Traditionally, RF antennas are designed using expensive, proprietary software packages, like XFDTD and HFSS \cite{hfss,xfdtd}.  Designs are fabricated by cutting and shaping metal with precision machine tools.  This technique is sometimes called \textit{subtractive manufacturing}, as opposed to \textit{additive manufacturing.}  Compared to \textit{additive manufacturing} techniques, subtractive manufacturing can be costly \cite{segoviaguerrero2025additive-29b,add_vs_sub}.  Open-source additive manufacturing boosts cost efficiency in both the design and fabrication of RF antennas, provided that the 3D printing filament has sufficient conductivity.

Development of new RF antenna designs requires exploration of conductor shapes that set the boundary conditions for radiated signals.  Developing new conductor shapes that meet design requirements has been aided by machine learning (ML) strategies \cite{gajbhiye2025comprehensive-fdb,goudos2016evolutionary-cc7,linden1999evolving-8c2}.  While mixing new ML algorithms into open CEM code is straightforward, incorporating them into proprietary software is often challenging and time-consuming.  Ideally, RF engineers would keep the functionality found in proprietary software to compute standard RF antenna metrics, while retaining the flexibility to incorporate new ML techniques.  Creating a design process based on open CEM tools achieves this goal.

Previously, J. C. Hanson (2021) demonstrated that the open-source MEEP software package may be used as an RF antenna design tool \cite{hanson2021broadband-cab}.  MEEP operates via the FDTD algorithm for Maxwell's Equations on a Yee lattice \cite{10.1016/j.cpc.2009.11.008}.  Specifically, the author of \cite{hanson2021broadband-cab} produced RF phased array designs in two and three dimensions using MEEP code.  The designs included broadband RF horn elements in the [0.5-5] GHz bandwidth.  Further, J. C. Hanson demonstrated that open-source CAD software may be used to create complex designs that can be 3D-printed \cite{meepcon_hanson}.

In this work, we present the first open-source broadband RF horn designed with MEEP and fabricated entirely from conductive 3D-printer filament.  In Sec. \ref{sec:mat_meth}, we explain the CEM design and open-source CAD (Secs. \ref{sec:mat_meth_1}-\ref{sec:mat_meth_2}), and fabrication technique (Sec. \ref{sec:mat_meth_3}).  In Sec. \ref{sec:results}, we show that the E-plane and H-plane radiation patterns (Sec. \ref{sec:results_1}), the VSWR (Sec. \ref{sec:results_2}), and the cross-polarization ratios (Sec. \ref{sec:results_3}) match predictions.  In Sec. \ref{sec:disc}, we discuss the limitations of our analysis, applications of the results, and future research directions.  In Sec. \ref{sec:conc}, we summarize our key results.

\section{Methods and Materials}
\label{sec:mat_meth}

The goal of this work was to use open CEM tools to design, model, and fabricate a linearly polarized, broadband RF horn antenna in the multi-GHz bandwidth.  We selected MEEP for our open CEM tools.  Over the past decade, MEEP has most often been used to design and model photonics systems \cite{hammond2022highperformance-057,majumder2017ultracompact-0fc}.  However, due to the scale-invariance of Maxwell's equations, MEEP may be used to model electromagnetic propagation with wavelengths at the cm-scale \cite{hanson2021broadband-cab,meepcon_hanson}.  Further, antenna designs created with open CAD tools like kLayout can be imported into MEEP.  Using parametric CAD design, we can specify antenna shapes using analytic functions that are then translated into kLayout and MEEP.  Designs in kLayout can also be translated into 3D printer files, ensuring we are simulating the same device we are fabricating.

Our open fabrication technique hinges on the existence of commerically available conductive 3D printer filament.  Multi3D\footnote{See \url{https://www.multi3dllc.com}} has shown that a copper-doped thermoplastic serves as a 3D printer filament with conductivities of $\approx 10^{-1}$ $\Omega$ cm, depending on the printing pattern.  Using a ruby-tipped 3D printing head and Electrifi filament from Multi3D, we can complete antenna prints on the scale of 16 cm in length in several hours.  The printed structures are fitted with RF connectors and tested with a network analyzer, standardized metal antennas, and custom mounts.  Figure \ref{fig:1} (left) contains a flow diagram for the design and fabrication process.

\subsection{CEM Design and Open-Source CAD}
\label{sec:mat_meth_1}

The kLayout design of the RF horn is shown in Figure \ref{fig:1}.  The open-source CEM design begins with choosing parametric design parameters for the exponential RF horn.  The horn \textit{cavity} is the rectangular volume where the coaxial cable attaches.  The horn \textit{surfaces} are the curved structures that connect the cavity to open space.  The \textit{opening} is the area where the \textit{surfaces} stop and radiation exits the horn.  Let the origin of an $xy$ coordinate system refer to the center of the outside of the cavity.  Let $c_l$ refer to the cavity length in the $x$-direction, $c_w$ refer to the cavity width in the $y$-direction, $s_l$ refer to the surface length in the $x$-direction, and $w$ refer to the opening width in the $y$-direction.  Using these variables, Equations \ref{eq:1} and \ref{eq:2} specify the shape of the RF horn surfaces in the $xy$-plane:

\begin{align}
f(x) &= \frac{c_w}{2}\exp(k(x-c_l)) \label{eq:1} \\
k &= (s_l-c_l)^{-1} \ln\left(\frac{w}{c_w}\right) \label{eq:2}
\end{align}

\noindent
The function $f(x)$ in Equation \ref{eq:1} describes the upper surface in Figure \ref{fig:1}, while $-f(x)$ describes the lower surface.  Table \ref{tab:1} contains the values for $c_l$, $c_w$, $s_l$, and $w$ corresponding to our first 3D printed RF horn.  The horn is designed to be linearly polarized in the $y$-direction.

\begin{figure}
\centering
\includegraphics[width=0.65\textwidth]{CAD_path.pdf}
\includegraphics[width=0.34\textwidth]{CAD_1.png}
\caption{\label{fig:1} (Left) Flow diagram for the open-source design and fabrication process.  (Right) The kLayout CAD design for the RF horn (top view), with a coaxial cable attached.  The $x$-direction is to the right, the $y$-direction is up, and the $z$-direction is out of the page.}
\end{figure}

\begin{table}
\caption{\label{tab:1} Design parameters for the RF horn antenna.}
\begin{tabularx}{\textwidth}{CCC}
\toprule
\textbf{Parameter} & \textbf{Variable Name} & \textbf{Value [cm]} \\
\midrule
Cavity Length & $c_l$ & 0.77 \\
Cavity Width & $c_w$ & 1.00 \\
Surface Length & $s_l$ & 16.52 \\
Opening Width & $w$ & 9.59 \\
\bottomrule
\end{tabularx}
\end{table}

In previous studies \cite{hanson2021broadband-cab}, antenna structures were instantiated as native MEEP objects.  For example, the following Python3 code with the Meep library imported as \verb+mp+ adds $N$ 2D exponential horns to the overall geometry via the \verb+mp.Block()+ method:

\small
\begin{verbatim}
for j in range(0,n_antenna):
    y = j*d_y+y0
    #Describe the cavity
    cav_back_location = mp.Vector3(x0,y)
    cav_location_pos = mp.Vector3(c_l/2.0+x0,c_w/2.0+y)
    cav_location_neg = mp.Vector3(c_l/2.0+x0,-c_w/2.0+y)
    backplate = mp.Block(backplate_size,center=cav_back_location,material=mp.metal)
    cav_upper = mp.Block(side_plate_size,center=cav_location_pos,material=mp.metal)
    cav_lower = mp.Block(side_plate_size,center=cav_location_neg,material=mp.metal)
    geometry.append(cav_upper)
    geometry.append(cav_lower)
    geometry.append(backplate)
    #Describe the horn surfaces
    size_upper = mp.Vector3(dx,thickness)
    size_lower = mp.Vector3(dx,thickness)
    for i in range(0,n_slices):
        center_upper = mp.Vector3(i*dx+c_l+x0,c_w/2*np.exp(k*(i*dx-c_l))+y)
        center_lower = mp.Vector3(i*dx+c_l+x0,-c_w/2*np.exp(k*(i*dx-c_l))+y)
        geometry.append(mp.Block(size_upper,center=center_upper,material=mp.metal))
        geometry.append(mp.Block(size_lower,center=center_lower,material=mp.metal))
\end{verbatim}
\normalsize

\noindent
With appropriate constants defined, this code creates a one-dimensional phased array of 2D RF horns with MEEP conductors.  There are two advantages to this \textit{parametric design}: is that it is fast and simple to code, and no other tools or programs outside of MEEP are required.  Structures can be quickly modified by exploring the parameter space defined by the constants in Table. \ref{tab:1}.  The disadvantage is that complex structures must be assembled from simple ones with little intuition or visualization.  Using open CAD programs like kLayout, however, allow the RF engineer to design antenna structures visually before importing them into MEEP.  By introducing open-source CAD, the same structures given to the CEM modeling can be given to the 3D printing system.

Ideally, the advantages of parametric design and open-source CAD should be preserved.  First, we print the vertices of the antenna structures from our Python3 code.  Second, we create polygons in kLayout, with initial lists of vertices.  Finally, we copy the vertices from Python3 code into the kLayout polygons.  This procedure results in 2D structures like those shown in Figure \ref{fig:1} (right).  Figure \ref{fig:1} (right) is a design of the RF horn in the $xy$-plane, and we give the design height and thickness in the $z$-direction after importing it to MEEP.  MEEP accepts the GDSII file format produced by kLayout, and GDSII files can be converted to STL files suitable for PrusaSlicer and other 3D printing software.

\subsubsection{Radiation Pattern Calculations}

We compute the antenna radiation pattern in MEEP using near-to-far field projection\footnote{See \url{https://meep.readthedocs.io/en/master/Python_Tutorials/Near_to_Far_Field_Spectra}}.  First, a \verb+NearToFarRegion+ is created to fully enclose the structure, similar to a Gaussian surface in classical electrodynamics.  Second, the \verb+get_farfield()+ function computes the $\mathbf{E}$ and $\mathbf{H}$ fields a distance $r$ from the origin.  In Figure \ref{fig:1}, the $xy$-plane is the $\mathbf{E}$-plane, and the $xz$-plane is the $\mathbf{H}$-plane.  We chose $r=10$ meters in our calculations, given the dimensions of the horn (Table. \ref{tab:1}) and $\approx 1$ cm wavelengths.  Third, we compute the normalized Poynting vector along the $\mathbf{E}$ and $\mathbf{H}$-planes.  The following code produces the normalized radiation pattern in dB, with MEEP as \verb+mp+, NumPy as \verb+np+, and \verb+proj_box+ as the \verb+NearToFarRegion+:

\small
\begin{verbatim}
def calculate_E_plane_rad_patt(sim,proj_box):
    r = 1000
    npts = 360
    E = np.zeros((npts,3),dtype=np.complex128)
    H = np.zeros((npts,3),dtype=np.complex128)
    angles = 2*np.pi/npts*np.arange(npts)
    for n in range(npts):
        x = r*np.cos(angles[n])
        y = r*np.sin(angles[n])
        ff = sim.get_farfield(proj_box,mp.Vector3(x,y,0))
        E[n,:] = [ff[j] for j in range(3)]
        H[n,:] = [ff[j+3] for j in range(3)]
    Px = np.real(np.conj(E[:,1])*H[:,2]-np.conj(E[:,2])*H[:,1])
    Py = np.real(np.conj(E[:,2])*H[:,0]-np.conj(E[:,0])*H[:,2])
    Pr = np.sqrt(np.square(Px) + np.square(Py))
    directivity = 10.0*np.log10(Pr/max(Pr))
    return (angles,directivity)
\end{verbatim}
\normalsize

\noindent
To obtain the radiation pattern from near-to-far field projection, the MEEP simulation must be run such that a sufficient amount of the radiation has passed through the \verb+NearToFarField+ region.

\subsubsection{VSWR Calculations}

\begin{figure}
\centering
\includegraphics[width=0.33\textwidth,trim=0cm 25cm 20cm 0cm,clip=true]{CAD_1.png}
\includegraphics[width=0.33\textwidth,trim=0cm 6cm 20cm 19cm,clip=true]{CAD_1.png}
\caption{\label{fig:2} (Left) The upper left region of \ref{fig:1}.  The purple region is the radiating source used for VSWR analysis, the red regions are dielectric material, and the light blue regions are conductors.  The combined structure represents a coaxial cable.  (Right)  The lower left region of \ref{fig:1}.  The coaxial cable reaches the horn cavity, shown in light blue.  The dark blue regions are the beginning of the horn surfaces.  The pink structure in the center is the radiation source used for radiation pattern analysis.}
\end{figure}

The voltage standing wave radio (VSWR) is a standard figure of merit in RF antenna characterization.  The VSWR may be expressed in terms of the complex reflection coefficient, $\Gamma$, between coaxial line and RF antenna input port:

\begin{equation}
VSWR = \frac{1+|\Gamma|}{1-|\Gamma|} \label{eq:3}
\end{equation}

\noindent
Our strategy to calculate the VSWR is to send a pulse signal down a model for a coaxial cable, and measure $\Gamma$.  This strategy is based on MEEP \verb+FluxRegion+ objects, similar to the \verb+NearToFarRegion+ objects in the previous section\footnote{See \url{https://meep.readthedocs.io/en/master/Python_Tutorials/GDSII_Import}.}.

Figure \ref{fig:2} (left) shows the upper left region of the RF horn CAD, depicting the radiation source within the structure representing the coaxial cable.  The coaxial cable model is built from outer conductors, dielectric materials, and an inner conductor with a radiating source polarized radially.  While the radiation pattern analysis uses a continuous-wave (CW) source, the source in our VSWR analysis is given a custom Gaussian signal with a broadband spectrum.  In the middle of the cable, a MEEP \verb+FluxRegion+ is instantiated.  Simular to a \verb+NearToFarRegion+, the \verb+FluxRegion+ records electromagnetic flux through a geometric region, simular to a Gaussian surface in classical electrodynamics.  Both MEEP classes can record flux at multiple frequencies.  In our CW radiation pattern analysis, we set the \verb+NearToFarRegion+ frequency to the source frequency.  In our VSWR analysis, we chose 1024 frequency bins centered on 5 GHz and passed them to the \verb+FluxRegion+.

The reflection coefficient $\Gamma$ is measured in two stages.  First, the MEEP simulation is run for a time corresponding to the length of the coaxial cable, accounting for the speed of propagation given the dielectric material.  The input flux is recorded, and the simulation is reset.  The simulation is run a second time, but the input flux is multiplied by $-1$ and pre-loaded into the \verb+FluxRegion+.  The simulation is run for a factor of 2 longer, so that any reflections from the RF antenna have time to reach the \verb+FluxRegion+ a second time.  The \verb+FluxRegion+ records the output flux, having already cancelled the input flux from the pre-loaded opposite input flux.  Thus, using one \verb+FluxRegion+, the ratio of output to input flux, $\Gamma$, can be calculated.  The VSWR ratio (Equation \ref{eq:3}) is formed and converted to dB.

\subsection{Open-Source CAD}
\label{sec:mat_meth_2}

things.

\subsection{Fabrication Technique}
\label{sec:mat_meth_3}

things.

\section{Results}
\label{sec:results}

This section gives results

\subsection{Radiation Patterns}
\label{sec:results_1}

The text continues here.

\subsection{The VSWR}
\label{sec:results_2}

The text continues here.

\subsection{Cross-Polarization Ratios}
\label{sec:results_3}

The text continues here.

\section{Discussion}
\label{sec:disc}

Authors should discuss the results and how they can be interpreted from the perspective of previous studies and of the working hypotheses.

\section{Conclusions}
\label{sec:conc}

This section is not mandatory, but can be added to the manuscript if the discussion is unusually long or complex.

\vspace{6pt} 

\authorcontributions{J. C. Hanson conceived of the open-source design process and produced the initial MEEP calculations that led to the RF horn design, including a previous publication in \textit{Electronics Journal}.  J. C. Hanson identified the conductive 3D printer filament with sufficient conductivity to fabricate RF antennas.  J. C. Hanson demonstrated that open-source CAD can be incorporated into the process.  Finally, J. C. Hanson produced fully three-dimensional CEM models using MEEP to compute the VSWR and radiation patterns for the fabricated designs.  A. Wildanger sourced materials, and imported CAD designs into 3D printing format.  A. Wildanger successfully completed 3D prints to produce the designs.  Together, J. C. Hanson and A. Wildanger collected data using the RF measurement tools.  J. C. Hanson used the data to show that the MEEP calculations match the lab measurements.  A. Wildanger explored new models to be printed using open-source and free CAD programs, including larger RF horn models.}

\funding{This research was funded by NAME OF FUNDER grant number XXX.}

\dataavailability{We encourage all authors of articles published in MDPI journals to share their research data. } 

\acknowledgments{Acknowledge Lisa Newton and Sal Johnston}

\conflictsofinterest{The authors declare no conflicts of interest.} 

\abbreviations{Abbreviations}{
The following abbreviations are used in this manuscript:
\\
\noindent 
\begin{tabular}{@{}ll}
CEM & Computational Electromagnetism \\
CW & Continuous Wave \\
MEEP & MIT Electromagnetic Equation Propagator \\
NEEC & Naval Engineering Education Consortium \\
NSWC Corona & Naval Surface Warfare Center, Corona Division \\
RF & Radio-frequency
\end{tabular}
}

%%%%%%%%%%%%%%%%%%%%%%%%%%%%%%%%%%%%%%%%%%
%% Optional
\appendixtitles{no} % Leave argument "no" if all appendix headings stay EMPTY (then no dot is printed after "Appendix A"). If the appendix sections contain a heading then change the argument to "yes".
\appendixstart
\appendix
\section[\appendixname~\thesection]{}
The appendix is an optional section that can contain details and data supplemental to the main text
\section[\appendixname~\thesection]{}
All appendix sections must be cited in the main text. 

\begin{adjustwidth}{-\extralength}{0cm}

\reftitle{References}

\bibliography{references}

\PublishersNote{}
\end{adjustwidth}
\end{document}