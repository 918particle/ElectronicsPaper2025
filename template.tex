\documentclass[electronics,article,submit,pdftex,moreauthors]{Definitions/mdpi} 
%=================================================================
% MDPI internal commands - do not modify
\firstpage{1} 
\makeatletter 
\setcounter{page}{\@firstpage} 
\makeatother
\pubvolume{1}
\issuenum{1}
\articlenumber{0}
\pubyear{2025}
\copyrightyear{2025}
%\externaleditor{Firstname Lastname} % More than 1 editor, please add `` and '' before the last editor name
\datereceived{ } 
\daterevised{ } % Comment out if no revised date
\dateaccepted{ } 
\datepublished{ } 
%\datecorrected{} % For corrected papers: "Corrected: XXX" date in the original paper.
%\dateretracted{} % For retracted papers: "Retracted: XXX" date in the original paper.
\hreflink{https://doi.org/} % If needed use \linebreak
%\doinum{}
%\pdfoutput=1 % Uncommented for upload to arXiv.org
%\CorrStatement{yes}  % For updates
%\longauthorlist{yes} % For many authors that exceed the left citation part
%\IsAssociation{yes} % For association journals

%=================================================================
\usepackage{graphicx}
%=================================================================

\Title{Broadband RF Phased Array Design with MEEP: A 3D-Printed Open-Source RF Horn in the multi-GHz Bandwidth}
\TitleCitation{Title}
\newcommand{\orcidauthorA}{0000-0003-4055-4553}
\Author{Jordan C. Hanson $^{1}$\orcidA{}*, Adam Wildanger$^{1}$}
\AuthorNames{Jordan C. Hanson, Adam Wildanger}
\AuthorCitation{Hanson, J. C.; Wildanger, A.}
\address[1]{$^{1}$ \quad Whittier College, Whittier, CA, USA (www.whittier.edu)}
\corres{Correspondence: jhanson2@whittier.edu}

\abstract{Radio-frequency (RF) antenna design traditionally proceeds in three phases.  First, the design performance is modeled with expensive and proprietary computational electromagnetism (CEM) software.  Second, the design is fabricated using intricate metal machining.  Third, the fabricated design is characterized using benchtop RF measurement tools.  The traditional process can include machine learning algorithms for optimization.  We have developed an open-source alternative process that utilizes the MIT Electromagnetic Equation Propagator (MEEP) CEM package for design and simulation, and 3D printing with conductive filament for fabrication.  Using this process, we designed and fabricated an exponentially curved RF horn antenna.  To characterize our design, we show that the E-plane and H-plane radiation patterns, the VSWR, and the cross-polarization ratios match our CEM calculations.  These results indicate that our design is a linearly polarized, broadband RF horn antenna in the [5.5-6] GHz bandwidth.  The bandwidth of our instrumenation is limited to 6 GHz, and our CEM calculations predict good performance above 6 GHz.  We therefore conclude the bandwidth of our printed design extends above 6 GHz.  Future work will include expanding to lower bandwidth by 3D printing larger antennas, and constructing a broadband RF phased array.}

% Keywords
\keyword{Computational Electromagnetism (CEM), Additive Manufacturing, MEEP, \\ RF Engineering, Open-Source Design}

\begin{document}

\section{Introduction}

Broadband RF antennas are ubiquitous tools within scientific instrumentation and communication applications.  Traditionally, RF antennas are designed using expensive, proprietary software packages, like XFDTD and HFSS \cite{hfss,xfdtd}.  Designs are fabricated by cutting and shaping metal with precision machine tools.  This technique is sometimes called \textit{subtractive manufacturing}, as opposed to \textit{additive manufacturing.}  Compared to \textit{additive manufacturing} techniques, subtractive manufacturing can be costly \cite{segoviaguerrero2025additive-29b,add_vs_sub}.  Open-source additive manufacturing boosts cost efficiency in both the design and fabrication of RF antennas, provided that the 3D printing filament has sufficient conductivity.

Development of new RF antenna designs requires exploration of conductor shapes that set the boundary conditions for radiated signals.  Developing new conductor shapes that meet design requirements has been aided by machine learning (ML) strategies \cite{gajbhiye2025comprehensive-fdb,goudos2016evolutionary-cc7,linden1999evolving-8c2}.  While mixing new ML algorithms into open CEM code is straightforward, incorporating them into proprietary software is often challenging and time-consuming.  Ideally, RF engineers would take advantage of the functionality of proprietary software to compute standard RF antenna metrics like radiation patterns and VSWR, while retaining the flexibility to incorporate new ML techniques.  Determining how to use open CEM tools to compute standard RF antenna metrics gives RF engineers both necessary functionality and flexibility.

Previously, J. C. Hanson (2021) demonstrated that the open-source MEEP software package may be used as an RF antenna design tool \cite{hanson2021broadband-cab}.  MEEP operates via the FDTD algorithm for Maxwell's Equations on a Yee lattice \cite{10.1016/j.cpc.2009.11.008}.  Specifically, the author of \cite{hanson2021broadband-cab} produced RF phased array designs in two and three dimensions using MEEP code.  The designs included broadband RF horn elements in the [0.5-5] GHz bandwidth.  Further, J. C. Hanson demonstrated that open-source CAD software may be used to create complex designs that can be 3D-printed \cite{meepcon_hanson}.

In this work, we present the first open-source broadband RF horn designed with MEEP and fabricated entirely from conductive 3D-printer filament.  In Sec. \ref{sec:mat_meth}, we explain the CEM design (Sec. \ref{sec:mat_meth_1}), open-source CAD (Sec. \ref{sec:mat_meth_2}), and fabrication technique (Sec. \ref{sec:mat_meth_3}).  In Sec. \ref{sec:results}, we show that the E-plane and H-plane radiation patterns (Sec. \ref{sec:results_1}), the VSWR (Sec. \ref{sec:results_2}), and the cross-polarization ratios (Sec. \ref{sec:results_3}) match our CEM calculations.  In Sec. \ref{sec:disc}, we discuss the limitations of our analysis, applications of the results, and future research directions.  In Sec. \ref{sec:conc}, we summarize our key results.

\section{Materials and Methods}
\label{sec:mat_meth}

Materials and Methods should be described with sufficient details to allow others to replicate and build on published results.

\subsection{CEM Design}
\label{sec:mat_meth_1}

things.

\subsection{Open-Source CAD}
\label{sec:mat_meth_2}

things.

\subsection{Fabrication Technique}
\label{sec:mat_meth_3}

things.

\section{Results}
\label{sec:results}

This section gives results

\subsection{Radiation Patterns}
\label{sec:results_1}

The text continues here.

\subsection{The VSWR}
\label{sec:results_2}

The text continues here.

\subsection{Cross-Polarization Ratios}
\label{sec:results_3}

The text continues here.

\subsection{Figures, Tables and Schemes}

All figures and tables should be cited in the main text as Figure~\ref{fig1}, Table~\ref{tab1}, etc.

\begin{figure}[H]
\includegraphics[width=4.0 cm]{Definitions/logo-mdpi}
\caption{This is a figure. Schemes follow the same formatting.\label{fig1}}
\end{figure}   
\unskip

\begin{table}[H] 
\caption{This is a table caption. Tables should be placed in the main text near to the first time they are~cited.\label{tab1}}
\begin{tabularx}{\textwidth}{CCC}
\toprule
\textbf{Title 1}	& \textbf{Title 2}	& \textbf{Title 3}\\
\midrule
Entry 1	& Data & Data\\
Entry 2	& Data & Data \textsuperscript{1}\\
\bottomrule
\end{tabularx}
\noindent{\footnotesize{\textsuperscript{1} Tables may have a footer.}}
\end{table}

The text continues here (Figure~\ref{fig2} and Table~\ref{tab2}).

\section{Discussion}
\label{sec:disc}

Authors should discuss the results and how they can be interpreted from the perspective of previous studies and of the working hypotheses.

\section{Conclusions}
\label{sec:conc}

This section is not mandatory, but can be added to the manuscript if the discussion is unusually long or complex.

\vspace{6pt} 

\authorcontributions{J. C. Hanson conceived of the open-source design process and produced the initial MEEP calculations that led to the RF horn design, including a previous publication in \textit{Electronics Journal}.  J. C. Hanson also identified the conductive 3D printer filament with sufficient conductivity to fabricate RF antennas.  J. C. Hanson also demonstrated that open-source CAD can be incorporated into the process.  Finally, J. C. Hanson produced fully three-dimensional CEM models using MEEP to compute the VSWR and radiation patterns for the fabricated designs.  A. Wildanger sourced materials, and imported CAD designs into 3D printing format.  A. Wildanger successfully completed 3D prints to produce the designs.  Together, J. C. Hanson and A. Wildanger collected data using the RF measurement tools.  J. C. Hanson used the data to show that the MEEP calculations match the lab measurements.}

\funding{This research was funded by NAME OF FUNDER grant number XXX.}

\dataavailability{We encourage all authors of articles published in MDPI journals to share their research data. In this section, please provide details regarding where data supporting reported results can be found, including links to publicly archived datasets analyzed or generated during the study. Where no new data were created, or where data is unavailable due to privacy or ethical restrictions, a statement is still required. Suggested Data Availability Statements are available in section ``MDPI Research Data Policies'' at \url{https://www.mdpi.com/ethics}.} 

\acknowledgments{In this section you can acknowledge any support given which is not covered by the author contribution or funding sections. This may include administrative and technical support, or donations in kind (e.g., materials used for experiments). Where GenAI has been used for purposes such as generating text, data, or graphics, or for study design, data collection, analysis, or interpretation of data, please add “During the preparation of this manuscript/study, the author(s) used [tool name, version information] for the purposes of [description of use]. The authors have reviewed and edited the output and take full responsibility for the content of this publication.”}

\conflictsofinterest{Declare conflicts of interest or state ``The authors declare no conflicts of interest.'' Authors must identify and declare any personal circumstances or interest that may be perceived as inappropriately influencing the representation or interpretation of reported research results. Any role of the funders in the design of the study; in the collection, analyses or interpretation of data; in the writing of the manuscript; or in the decision to publish the results must be declared in this section. If there is no role, please state ``The funders had no role in the design of the study; in the collection, analyses, or interpretation of data; in the writing of the manuscript; or in the decision to publish the results''.} 

\abbreviations{Abbreviations}{
The following abbreviations are used in this manuscript:
\\

\noindent 
\begin{tabular}{@{}ll}
MDPI & Multidisciplinary Digital Publishing Institute\\
DOAJ & Directory of open access journals
\end{tabular}
}

%%%%%%%%%%%%%%%%%%%%%%%%%%%%%%%%%%%%%%%%%%
%% Optional
\appendixtitles{no} % Leave argument "no" if all appendix headings stay EMPTY (then no dot is printed after "Appendix A"). If the appendix sections contain a heading then change the argument to "yes".
\appendixstart
\appendix
\section[\appendixname~\thesection]{}
\subsection[\appendixname~\thesubsection]{}
The appendix is an optional section that can contain details and data supplemental to the main text---for example, explanations of experimental details that would disrupt the flow of the main text but nonetheless remain crucial to understanding and reproducing the research shown; figures of replicates for experiments of which representative data are shown in the main text can be added here if brief, or as Supplementary Data. Mathematical proofs of results not central to the paper can be added as an appendix.

\begin{table}[H] 
\caption{This is a table caption.\label{tab5}}
%\newcolumntype{C}{>{\centering\arraybackslash}X}
\begin{tabularx}{\textwidth}{CCC}
\toprule
\textbf{Title 1}	& \textbf{Title 2}	& \textbf{Title 3}\\
\midrule
Entry 1		& Data			& Data\\
Entry 2		& Data			& Data\\
\bottomrule
\end{tabularx}
\end{table}

\section[\appendixname~\thesection]{}
All appendix sections must be cited in the main text. In the appendices, Figures, Tables, etc. should be labeled, starting with ``A''---e.g., Figure A1, Figure A2, etc.

\begin{adjustwidth}{-\extralength}{0cm}

\reftitle{References}

\bibliography{references}

\PublishersNote{}
\end{adjustwidth}
\end{document}

